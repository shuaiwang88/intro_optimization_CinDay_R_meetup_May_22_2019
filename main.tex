\documentclass[xcolor=dvipsnames]{beamer}
\usepackage{xcolor}

\usepackage[utf8]{inputenc}
\usepackage{multicol}
\usepackage{utopia} %font utopia imported

\usetheme{Madrid}
% \usetheme[width=1cm]{PaloAlto} 
\definecolor{RBlue}{RGB}{44, 108, 188} 
\usecolortheme[named=RBlue]{structure}
%------------------------------------------------------------
%This block of code defines the information to appear in the
%Title page
\title[]{Introduction of Mathematical Optimization for R users!}

\subtitle{library(rcbc) \& minizinc}

\author[]{Shuai Wang \and Eugene Pyatigorsky}

\institute[] % (optional)
{Kroger/84.51 Operations Research}

\date[] % (optional)
{CinDay R User Meetup, May 22 2019}

\logo{\includegraphics[height=1cm]{CinDayR.png}}

%End of title page configuration block
%------------------------------------------------------------



%------------------------------------------------------------
%The next block of commands puts the table of contents at the 
%beginning of each section and highlights the current section:

\AtBeginSection[]
{
  \begin{frame}
    \frametitle{Table of Contents}
    \tableofcontents[currentsection]
  \end{frame}
}
%------------------------------------------------------------


\begin{document}

%The next statement creates the title page.
\frame{\titlepage}


%---------------------------------------------------------
%This block of code is for the table of contents after
%the title page
\begin{frame}
\frametitle{Table of Contents}
\tableofcontents
\end{frame}
%---------------------------------------------------------


\section{Mathematical Optimization}

%---------------------------------------------------------
\begin{frame}
\frametitle{What's mathematical optimization anyway?}
“Optimization” comes from the same root as “optimal”, which means best. When you
optimize something, you are “making it best”.


But “best” can vary. If you’re a football player, you might want to maximize your
running yards, and also minimize your fumbles. Both maximizing and minimizing are types
of optimization problems.
\end{frame}

%---------------------------------------------------------


%---------------------------------------------------------
%Example of the \pause command
\begin{frame}
\frametitle{Mathematical Optimization in the “Real World”}
Mathematical Optimization is a branch of applied mathematics which is useful in many different fields. Here are a few examples:
 \begin{multicols}{2}
    \begin{itemize}
        \item Manufacturing
        \item Production
        \item Inventory control
        \item Transportation
        \item Scheduling
        \item Networks
        \item Finance
        \item Economics
        \item Control engineering
        \item Marketing
        \item Policy Modeling
        \item Mechanics
    \end{itemize}
    \end{multicols}

\end{frame}


\begin{frame}{Optimization Model Components}
Your basic optimization problem consists of: 
\begin{enumerate}
    \item The objective function, {\color{red} f(x)}, which is the output you’re trying to maximize or minimize. e.g. maximize the gross profit margin; minimize travel distance of a pizza delivery car. 
    
    \item  Variables,  {\color{red}$x_1, x_2, x_3$} and so on, which are the inputs – things you can control. 
    
    \item Constraints, which are equations that place limits on how big or small some variables can get. e.g. The pizza delivery should be on time.
    
    
\end{enumerate}


\end{frame}


\begin{frame}{Optimization Example}
    A football coach is planning practices for his running backs.
\begin{itemize}
    \item His main goal is to maximize running yards – this will become his
    {\color{red} objective function} .
    
    \item He can make his athletes spend practice time in the weight room; running 
    sprints; or practicing ball protection. The amount of time spent on each is a 
    {\color{red} variable}.
    
    \item  However, there are limits to the total amount of time he has. Also, if he
    completely sacrifices ball protection he may see running yards go up, but also 
    fumbles, so he may place an upper limit on the amount of fumbles he considers 
    acceptable. These are {\color{red} constraints}.
\end{itemize}

Note that the variables influence the objective function and the constraints place limits on the domain of the variables.
\end{frame}





\section{Second section}

%---------------------------------------------------------
%Highlighting text
\begin{frame}
\frametitle{Sample frame title}

In this slide, some important text will be
\alert{highlighted} because it's important.
Please, don't abuse it.

\begin{block}{Remark}
Sample text
\end{block}

\begin{alertblock}{Important theorem}
Sample text in red box
\end{alertblock}

\begin{examples}
Sample text in green box. The title of the block is ``Examples".
\end{examples}
\end{frame}
%---------------------------------------------------------


%---------------------------------------------------------
%Two columns
\begin{frame}
\frametitle{Two-column slide}

\begin{columns}

\column{0.5\textwidth}
This is a text in first column.
$$E=mc^2$$
\begin{itemize}
\item First item
\item Second item
\end{itemize}

\column{0.5\textwidth}
This text will be in the second column
and on a second tought this is a nice looking
layout in some cases.
\end{columns}
\end{frame}
%---------------------------------------------------------


\end{document}